\documentclass[12pt]{article}
\usepackage{cite}
\usepackage{color}
\usepackage{graphicx}
\usepackage{caption}
\usepackage{subcaption}
\usepackage{parskip}

\setlength{\parskip}{5pt}
\setlength{\parindent}{0pt}

\title{LC/MS Chromatogram Smoothing With Generative Adversarial Networks}

\author{Gabriel Ferns(CS Senior) , Max Horowitz-Gelb(CS Senior)}

\begin{document}

\maketitle

\abstract{
Liquid Chromatography Mass Spectrometry is a powerful tool for identifying and quantifying proteins in a complex sample. High throughput methods of LC/MS such as DIA allow for analysis of samples with thousands of proteins. But this increase in throughput comes at the cost of a decrease in signal quality. This gives incentive for post extraction methods to clean and smooth chromatogram signal data. Here , using a new and powerful deep learning framework known as GANs \cite{GAN}, we attempt to smooth chromatograms, without distortion of the true underlying quantifiable data. 
\color{red}
Highlight results here
\color{black}
}

\section{Introduction}
DIA \cite{DIA} is a method of LC/MS for quanitifying thousands of proteins. In order to to do analyze thousands of proteins at the same time, DIA methods have one cycling through thousands of different m/z windows, where each window is used to quantify a single peptide. Because we are analyzing so many windows at each time, the dwell time, or time in which signal is extracted, must be very short. Because of this, the quality of the data extracted can be quite poor. An alternative to DIA is SRM, or Selected Reaction Monitoring,\cite{SRM}. SRM is not good for quantifying more than a handful of peptides at a time, but the data generated is of far superior quality to DIA. It would be nice if we could keep the high throughput of DIA while also having data with the same quality as SRM. Then one could analyze thousands of proteins at the same time and get clean reliable data for quantification.

The quality of SRM data in comparison to DIA data suggests that perhaps a generative model trained on clean SRM data, could remove noise and smooth DIA data without distorting the actual quantifiable information. To test this hypothesis we chose to use a GAN as our generative model. GAN neural networks have been shown as powerful tools for image processing particularly for removal of noise and upscaling of images \cite{SRGAN}\cite{DERAIN}. We applied similar methods to chromatogram data by training a GAN on clean SRM data with simulated Gaussian noise. We then tested our GAN on a dataset of real DIA data.

\color{red}
Include results here
\color{black}

\begin{figure}
\centering
\begin{subfigure}{.5\textwidth}
  \centering
  \includegraphics[width=1\linewidth]{SRM_example}
  \caption{SRM Chromatogram}
  \label{fig:sub1}
\end{subfigure}%
\begin{subfigure}{.5\textwidth}
  \centering
  \includegraphics[width=1\linewidth]{DIA_example}
  \caption{DIA Chromatogram}
  \label{fig:sub2}
\end{subfigure}
\caption{Here we show two chromatograms. As one can see the noise and the smoothness is much better in the chromatogram coming from an SRM dataset.}
\label{fig:test}
\end{figure}

\section{Brief LC/MS Primer}
LC/MS identifies and quantifies proteins by separating them by their hydrophobicity and mass. In general it is difficult to quantify whole proteins so we instead use peptides which are sub-sequences of a protein. These peptides come off of a liquid column at a specific time which is dependent on the hydrophobicity of the peptide.This time is known as the peptide's Retention Time. Once the peptide comes off the column it is ionized and then becomes what is called a precursor ion. When precursor ions go into the mass spectrometer they are filtered by the ratio of their mass to charge. A precursor ion's mass to charge along with its retention time are used as a two fold filtration process for unique quantification of peptides. 

At this point one could start quantifying the peptide, but in certain methods, like DIA, the precursor is then blasted it with an ion spray. This ion spray causes the precursor to fragment into two pieces. One of these pieces, known as a transition ion, becomes charged and may be detected by the machine. The ratio of expected number of transition ions for each possible fragment that could occur is known via empirical evidence. Because of this the relative ion intensity of each transition ion is useful for identification of the peptide.

\begin{figure}
\centering
\begin{subfigure}{.5\textwidth}
  \centering
  \includegraphics[width=1\linewidth]{precursor}
  \caption{Precursor Chromatogram}
  \label{fig:sub1}
\end{subfigure}%
\begin{subfigure}{.5\textwidth}
  \centering
  \includegraphics[width=1\linewidth]{transitions}
  \caption{Transition Chromatograms(Peak Group)}
  \label{fig:sub2}
\end{subfigure}
\caption{Here we show the difference between precursor and transition ion chromatograms. The precursor chromatogram contains a single peak, while the transition chromatogram shows a set of peaks coming from all the different transitions or fragment ions. This set of transition chromatograms coming from the same peptide is known as a peak group.}
\label{fig:test}
\end{figure}


\section{Task Definition}
The goal of this project is to train a GAN so that it can take a peak group and return the same peak group after being smoothed and denoised.  
The GAN smooths and denoises each transition separately. The network takes as input the chromatogram from a single transition as well as an average signal from all the other transitions coming from the same peak group.  The reason this is done is that as one can see from figure 2, each transition chromatogram has a similar shape to all the other chromatograms in the peak group. So having an average of the other chromatograms is a clue into the true shape of the transition chromatogram we are trying to smooth. 

It then outputs a smoothed signal of the single transition. To smooth an entire peak group we apply the GAN network to each of the peak group's transitions and compile all the outputs together into a single smoothed peak group. 

\color{red}
Gabe describe structure of the Gans net. 
\color{black}
\section{Algorithm Definition}
\subsection{Chromatogram Preprocessing}
Before we can use the chromatogram 
\section{Methodology}
\section{Results}
\section{Discussion}
\section{Related Work}
\section{Future Work}
\section{Conclusion}

\bibliography{writeup}{}
\bibliographystyle{plain}

\end{document}